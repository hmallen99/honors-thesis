\documentclass[12pt]{article}
\usepackage[utf8]{inputenc}
\usepackage[margin=0.5in]{geometry}
\usepackage{bm}
\usepackage{graphicx}
\usepackage[rightcaption]{sidecap}
\usepackage{hyperref}

\title{ML Documentation}

\author{Henry Allen}

\begin{document}

\maketitle
\section{Machine Learning Analysis}
\subsection{Machine Learning Methods}
Analysis was performed in Python using the Keras library from TensorFlow, decoding libraries
from MNE-Python, and the sklearn library. Data was split into training sets of 500 trials for
each subject with 100 trials of test trials. Trials were time-binned from 0-0.4 seconds at 0.025
second intervals to account for the 40hz MEG data. The data is split into classes based on angle.
For example, if we wanted 4 classes, class 0 corresponds to gabors oriented from 0-45 degrees,
class 1 from 45-90, class 2 from 90-135, class 3 from 135-180. An issue is immediately apparent from
these class divisions. A gabor oriented at 179 degrees is marked as class 3 despite being very similar
to a gabor marked in class 1, at 1 degree for example. To account for this, we can use a few tactics:
(1) using many classes to minimize the inaccuracy of any one individual class, (2) split the first class
between 180 degrees and 0 degrees, or (3) develop a cost function that takes the cosine encoding into account. 
Models were trained for each subject to account
for individual Model accuracy was calculated with 5-fold cross-validation and evaluation accuracy
on the test dataset. Model accuracy was then compared with the results of a permutation test, in
which we shuffled around the training and test labels of the dataset to calculate a baseline accuracy
for a naive model. This test essentially compares our models to a chance accuracy. 

\subsection{Sliding Models}
\subsubsection{Logistic Regression}


\subsubsection{Neural Network}

\subsection{Recurrent Models}

\subsection{Serial Dependence Model}
    
\end{document}